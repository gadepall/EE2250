%\subsection{Karush Kuhn-Tucker Conditions}

\begin{problem}
%
\label{ch3_convex_ch2}
The problem
\begin{equation}
\min_{\mbf{x}} x_{11} + x_{12}
\end{equation}
%	
with constraints
\begin{align}
x_{11} + x_{22} &= 1 \\	
\begin{pmatrix}
x_{11} & x_{12} \\
x_{12} & x_{22}
\end{pmatrix} & \succeq 0 \quad  \brak{\text{$\succeq$ means positive definite}}
\end{align}
%
is known as a semi-definite program.  Find a numerical solution to this problem. Compare with the solution 
in problem  \ref{convex_sdp_eqiv}.
\end{problem}
\solution The {\em cvxopt} solver needs to be used in order to find a numerical solution.  For this, the given problem has to be reformulated as
\begin{align}
&\min_{\mbf{x}}  
\begin{pmatrix}
1 & 1 & 0
\end{pmatrix}
\begin{pmatrix}
x_{11} 
\\
x_{12}
\\
x_{22}
\end{pmatrix}
\quad \text{s.t}
\\
&
\begin{pmatrix}
1 & 0 & 1
\end{pmatrix}
\begin{pmatrix}
x_{11} 
\\
x_{12}
\\
x_{22}
\end{pmatrix}
=1
\\
&x_{11}
\begin{pmatrix}
-1 & 0 
\\
0 & 0
\end{pmatrix}
+
x_{12}
\begin{pmatrix}
0 & -1
\\
-1 & 0
\end{pmatrix}
+x_{22}
\begin{pmatrix}
0 & 0 
\\
0 & -1
\end{pmatrix}
\preceq 
\begin{pmatrix}
0 & 0 
\\
0 & 0
\end{pmatrix}.
\end{align}
%
The following script provides the solution to this problem.
\lstinputlisting{./chapter3/codes/3.1.py}
%
\begin{problem}
	Show that problem \ref{ch3_convex_ch2} is equivalent to problem \ref{convex_sdp_eqiv}.
\end{problem}
%
\begin{problem}
Minimize 
\begin{equation}
-x_{11} - 2x_{12} - 5x_{22}
\end{equation}
subject to
\begin{align}
\label{ch3_lin_mat_ineq_const}
2x_{11} + 3x_{12} + x_{22} &= 7 \\
x_{11} + x_{12} &\geq 1 \\
x_{11}, x_{12}, x_{22} &\geq 0 \\
\begin{pmatrix}
x_{11} & x_{12} \\
x_{12} & x_{22}
\end{pmatrix} & \succeq 0 
\end{align}
using {\em cvxopt}.
\end{problem}
%\solution
%In this problem, there is an SDP inequality and several linear inequalities.  The linear inequalities can be combined to obtain the matrix inequality
%%
%\begin{equation}
%\label{ch3_lin_mat_ineq}
%\begin{pmatrix}
%x_{11} + x_{12}- 1 & 0  & 0 & 0\\
%0 & x_{11} & 0 & 0
%\\
%0 & 0 & x_{12} &  0
%\\
 %0 & 0 & 0 & x_{22} 
%\end{pmatrix}
 %\succeq 0 
%\end{equation}
%%
%\eqref{ch3_lin_mat_ineq} can be combined with the matrix inequality in \eqref{ch3_lin_mat_ineq_const} to obtain the composite SDP
%\begin{equation}
%\label{ch3_lin_mat_sdp_ineq}
%\begin{pmatrix}
%\begin{matrix}
%x_{11} + x_{12}- 1 & 0  & 0 & 0\\
%0 & x_{11} & 0 & 0
%\\
%0 & 0 & x_{12} &  0
%\\
 %0 & 0 & 0 & x_{22} 
%\end{matrix}
%& \mbf{0}
%\\
%\mbf{0} & \begin{matrix}
%x_{11} & x_{12} \\
%x_{12} & x_{22}
%\end{matrix}
%\end{pmatrix} 
 %\succeq 0 
%\end{equation}
%%
%For using  {\em cvxopt}, the SDP in \eqref{ch3_lin_mat_sdp_ineq} can be expressed as
%%
%\begin{equation}
 %x_{11}F_{0} + x_{12}F_1+x_{22}F_{2}\succeq B ,
%\end{equation}
%%
%where
%%
%\begin{align}
%F_{0} = 
%\begin{pmatrix}
%\begin{matrix}
%1 &
%\\
%& 1
%\end{matrix}
%& & \bigzero
%\\
%& 
%\begin{matrix}
%0 &
%\\
%& 0
%\end{matrix}
%&
%\\
%\bigzero& & 
 %\begin{matrix}
%1 &  \\
 %& 0
%\end{matrix}
%\end{pmatrix} 
%\\
%F_{1} = 
%\begin{pmatrix}
%\begin{matrix}
%1 &
%\\
%& 0
%\end{matrix}
%& & \bigzero
%\\
%& 
%\begin{matrix}
%1 &
%\\
%& 0
%\end{matrix}
%&
%\\
%\bigzero& & 
 %\begin{matrix}
%0 & 1 \\
%1 & 0
%\end{matrix}
%\end{pmatrix} 
%\\
%F_2 = 
%\begin{pmatrix}
%\begin{matrix}
%0 &
%\\
%& 0
%\end{matrix}
%& & \bigzero
%\\
%& 
%\begin{matrix}
%0 &
%\\
%& 1
%\end{matrix}
%&
%\\
%\bigzero& & 
 %\begin{matrix}
%0 &  \\
 %& 1
%\end{matrix}
%\end{pmatrix} 
%\end{align}
%and
%\begin{align}
%B=
%\begin{pmatrix}
%\begin{matrix}
%1 &
%\\
%& 0
%\end{matrix}
%& & \bigzero
%\\
%& 
%\begin{matrix}
%0 &
%\\
%& 0
%\end{matrix}
%&
%\\
%\bigzero& & 
 %\begin{matrix}
%0 &  \\
 %&0
%\end{matrix}
%\end{pmatrix} 
%\end{align}
%%
\begin{problem}
	Repeat the above exercise by converting the problem into a convex optimization problem in two variables and using graphical plots.  
\end{problem}
\begin{problem}
	Solve the above problem using the KKT conditions.  Comment.
\end{problem}
	
	
	


